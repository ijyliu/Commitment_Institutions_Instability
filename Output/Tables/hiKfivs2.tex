{
\def\sym#1{\ifmmode^{#1}\else\(^{#1}\)\fi}
\begin{tabular}{l*{5}{c}}
\toprule
                &\multicolumn{1}{c}{(1)}&\multicolumn{1}{c}{(2)}&\multicolumn{1}{c}{(3)}&\multicolumn{1}{c}{(4)}&\multicolumn{1}{c}{(5)}\\
                &\multicolumn{1}{c}{Head of Govt. Turnover}&\multicolumn{1}{c}{Head of State Turnover}&\multicolumn{1}{c}{Lower House Turnover}&\multicolumn{1}{c}{WB Political Stability (Absence of Violence)}&\multicolumn{1}{c}{Instability Event Indicator}\\
\midrule
(Lack of)       &    1.416         &   -1.320         &    2.343         &    15.39\sym{**} &    4.248         \\
Irregular CB Governor Turnover (higher = more de facto CBI)&   (1.02)         &  (-1.07)         &   (1.28)         &   (3.29)         &   (1.32)         \\
\addlinespace
Exchange Rate   &   0.0181         &  -0.0133         &   0.0764         &   0.0870         &   0.0334         \\
Classification (RR inverted, higher = more fixed)&   (0.59)         &  (-0.40)         &   (1.89)         &   (0.91)         &   (0.69)         \\
\addlinespace
Constant        &   -0.705         &    1.777         &   -1.954         &   -14.51\sym{***}&   -3.497         \\
                &  (-0.56)         &   (1.69)         &  (-1.18)         &  (-3.36)         &  (-1.17)         \\
\midrule
Observations    &      571         &      570         &      476         &     1320         &     1001         \\
\bottomrule
\multicolumn{6}{l}{\footnotesize \textit{t} statistics in parentheses}\\
\multicolumn{6}{l}{\footnotesize \sym{*} \(p<0.05\), \sym{**} \(p<0.01\), \sym{***} \(p<0.001\)}\\
\end{tabular}
}
