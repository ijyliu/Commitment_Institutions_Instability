\documentclass{beamer}
% Class options include: notes, notesonly, handout, trans,
%                        hidesubsections, shadesubsections,
%                        inrow, blue, red, grey, brown

% Theme for beamer presentation.
\usepackage{beamerthemesplit} 
% Other themes include: beamerthemebars, beamerthemelined, 
%                       beamerthemetree, beamerthemetreebars  

\title[Commitment Institutions and Instability]{Commitment Institutions and Electoral and Political Instability}
\subtitle{A Reduced-Form Approach}
\author{Isaac Liu}
\date{\today}

\usepackage{graphicx} % http://ctan.org/pkg/graphicx
\usepackage{amsmath}
\usepackage{geometry}
\usepackage{amsfonts}
\usepackage[english]{babel}
\usepackage{amssymb}
\usepackage{graphicx}
\usepackage{float}
\usepackage{hyperref}
\usepackage{multirow}
\usepackage{pdflscape}
\usepackage{caption}
\usepackage{pdflscape}
\usepackage{outlines}
\usepackage{subcaption}

\usepackage{framed}

\usepackage{tabularx, booktabs}

\usepackage{standalone}

\usepackage[autostyle, english = american]{csquotes}
\MakeOuterQuote{"}

\usepackage{comment}

\hypersetup{
    colorlinks=true,
    linkcolor=white,
    filecolor=black,      
    urlcolor=black,
}

\setbeamertemplate{navigation symbols}{}

\begin{document}

    \begin{frame}
        \titlepage
    \end{frame}

    \begin{frame}
        \frametitle{\small Do the commitment institutions of central bank independence and fixed exchange rates affect electoral and political instability?}
        \begin{itemize}
            \item Net Welfare Benefits
            \begin{itemize}
                \item Inflation Time Inconsistency
                \item Political efficacy, access to capital
                \item Economic Voting, Increased Stability
            \end{itemize}
            \item Political Business Cycles
            \begin{itemize}
                \item Inability to manipulate economy or satisfy partisans
                \item Monetary (perhaps fiscal) policy
                \item Economic voting, Decreased Stability
            \end{itemize}
        \end{itemize}

        \begin{figure}[h]
    
            \begin{subfigure}[h]{0.475\textwidth}
                \centering
                \includegraphics[width=0.95\hsize]{../../Output/Figures/Fire_JP_Headline.png} 
            \end{subfigure}
            \hfill
            \begin{subfigure}[h]{0.475\textwidth}
                \centering
                \includegraphics[width=0.95\hsize]{../../Output/Figures/Trump_and_JP.png}
            \end{subfigure}
    
        \end{figure}

    \end{frame}

    \begin{frame}
        \frametitle{Theoretical Mechanisms}
        \begin{figure}[h!]
            \centering
            \caption{Effects of Limiting Institutions on Instability}
                \begin{subfigure}{.5\textwidth}
                    \centering
                    \includegraphics[width=\linewidth,keepaspectratio=true]{../../Output/Figures/Social_Welfare_Model.png}
                    \caption{Welfare Model}
                \end{subfigure}%
                \begin{subfigure}{.5\textwidth}
                    \centering
                    \includegraphics[width=\linewidth,keepaspectratio=true]{../../Output/Figures/Political_Business_Cycle_Model.png}
                    \caption{Political Business Cycle Model}
                \end{subfigure}
        \end{figure}
    \end{frame}

    \begin{frame}
        \frametitle{Literature}
        \begin{itemize}
            \item Bernhard and Leblang (2002)
            \begin{itemize}
                \item OLS, 16 parliamentary democracies since 1970s
                \item CBI increases cabinet duration by 3mos, Fixed rates by 5mos, especially with open trade and capital account
            \end{itemize}
            \item Clark, Golder, and Poast (2013)
            \begin{itemize}
                \item Survival Analysis, 19 OECD countries since 1970s
                \item Both institutions increase leader survival but only after 7y in office
            \end{itemize}
            \item Contribution:
            \begin{itemize}
                \item Far larger dataset including non/semi-democracies
                \item More consideration of endogeneity: choice of institutions based on stability consideration, de jure independence
                \item Political, not just electoral stability (coups, civil wars, etc), consideration for specific governmental positions
            \end{itemize}
        \end{itemize}
    \end{frame}

    \begin{frame}
        \frametitle{Data}
        \begin{itemize}
            \item Panel of 192 countries, 1970-2016
            \item Varieties of Democracy
            \begin{itemize}
                \item V2elturnhos, v2eltturnhog, v2eltvrig
                \item 0 for same individual (no turnover), 1 for same party or coalition (half turnover), 2 for new party \& ind. (full turnover)
                \item WGI Political Violence (neg = unstable)
                \item Instability Event- coup, civil war, internal conflict
            \end{itemize}
            \item Garriga (Cukierman, Webb, Neyapti)- de jure CBI
            \item Dreher et al.- Irregular turnover of governor- de facto CBI
            \item Reinhart, Rogoff Exchange Rates: 16 categories (higher = float)
        \end{itemize}
    \end{frame}

    \begin{frame}
        \frametitle{Results}
        \begin{itemize}
            \item Separate regressions (bad control problem)
            \item FEs, clustered SEs
            \item De Jure CBI and more instability: PBCs
            \item De Facto CBI (high irregular turnover) and less lower chamber turnover
            \item Fixed rate and less HOS turnover
            \item Welfare Benefits of De Facto CBI, Fixed Rates?
        \end{itemize}
    \end{frame}

    \begin{frame}
        \frametitle{Fixed Effects Regression with Clustered Standard Errors}
        {
            \let\oldcentering\centering
            \renewcommand\centering{\tiny\oldcentering}
            De Jure CBI, Fixed Effects Regression with Clustered Standard Errors \label{multIndFEDJ}
                    &miFEDJ_v2elturnhog&miFEDJ_v2elturnhos&miFEDJ_v2eltvrig&miFEDJ_e_wbgi_pve&miFEDJ_instabEvent\\
                    &           b&           b&           b&           b&           b\\
De Jure CBI (CNW Index)&    .2758142&    .3034033&    .3894757&   -.4167251&    1.000205\\
Exchange Rate Classification (RR inverted, higher = more fixed)&   -.0119552&   -.0207277&   -.0061496&    .0106214&    .0069002\\
_cons               &    .6180427&    .3897243&    .5350789&    .0282809&   -.1134885\\

        }
    \end{frame}

    \begin{frame}
        \frametitle{Fixed Effects Regression with Clustered Standard Errors}
        {
            \let\oldcentering\centering
            \renewcommand\centering{\tiny\oldcentering}
            \begin{table}[htbp]\centering
\def\sym#1{\ifmmode^{#1}\else\(^{#1}\)\fi}
\caption{De Facto CBI, Fixed Effects Regression with Clustered Standard Errors \label{multIndFEDF}}
\begin{tabular*}{\linewidth}{@{\hskip\tabcolsep\extracolsep\fill}l*{5}{c}}
\hline\hline
                &\multicolumn{1}{c}{(1)}&\multicolumn{1}{c}{(2)}&\multicolumn{1}{c}{(3)}&\multicolumn{1}{c}{(4)}&\multicolumn{1}{c}{(5)}\\
                &\multicolumn{1}{c}{Head of Govt. Turnover}&\multicolumn{1}{c}{Head of State Turnover}&\multicolumn{1}{c}{Lower House Turnover}&\multicolumn{1}{c}{WB Political Stability (Absence of Violence)}&\multicolumn{1}{c}{Instability Event Indicator}\\
\hline
(Lack of) Irregular CB Governor Turnover (higher = more de facto CBI)&   -0.117         &  -0.0512         &   -0.211\sym{**} &  0.00955         &   0.0244         \\
                &  (-1.68)         &  (-0.81)         &  (-2.81)         &   (0.36)         &   (1.36)         \\
[1em]
Exchange Rate Classification (RR inverted, higher = more fixed)& -0.00548         &  -0.0117\sym{*}  &  0.00444         &   0.0153\sym{*}  &   0.0128\sym{**} \\
                &  (-0.82)         &  (-2.06)         &   (0.53)         &   (2.08)         &   (2.73)         \\
[1em]
Constant        &    0.805\sym{***}&    0.521\sym{***}&    0.865\sym{***}&   -0.247\sym{***}&    0.261\sym{***}\\
                &   (9.91)         &   (7.75)         &   (9.43)         &  (-3.54)         &   (6.77)         \\
\hline
Observations    &     1651         &     1651         &     1334         &     2669         &     4491         \\
\hline\hline
\multicolumn{6}{l}{\footnotesize \textit{t} statistics in parentheses}\\
\multicolumn{6}{l}{\footnotesize \sym{*} \(p<0.05\), \sym{**} \(p<0.01\), \sym{***} \(p<0.001\)}\\
\end{tabular*}
\end{table}

        }
    \end{frame}

    \begin{frame}
        \frametitle{Ordered Logit (Mean Marginal Effects)}
        \begin{itemize}
            \item Nothing changes in terms of significance, except for fixed Erates and HOG
            \item xtologit; random effects
        \end{itemize}
    \end{frame}

    \begin{frame}
        \frametitle{Ordered Logit Mean Marginal Effects}
        {
            \let\oldcentering\centering
            \renewcommand\centering{\tiny\oldcentering}
                                &ordLogv2elturnhogDJ&ordLogv2elturnhosDJ&ordLogv2eltvrigDJ\\
                    &           b&           b&           b\\
De Jure CBI (CNW Index)&            &            &            \\
1._predict          &   -.1458306&   -.2079982&   -.3162625\\
2._predict          &    .0152021&    .0389939&    .0979976\\
3._predict          &    .1306285&    .1690043&    .2182649\\
Exchange Rate Classification (RR inverted, higher = more fixed)&            &            &            \\
1._predict          &    .0079217&    .0089644&    .0039218\\
2._predict          &   -.0008258&   -.0016806&   -.0012152\\
3._predict          &   -.0070959&   -.0072839&   -.0027066\\

        }
    \end{frame}

    \begin{frame}
        \frametitle{Ordered Logit Mean Marginal Effects}
        {
            \let\oldcentering\centering
            \renewcommand\centering{\tiny\oldcentering}
            \begin{table}[htbp]\centering
\def\sym#1{\ifmmode^{#1}\else\(^{#1}\)\fi}
\caption{De Facto CBI, Mean Marginal Effects, Ordered Logit Panel Regression, Random Effects, Clustered Standard Errors \label{ordLogDF}}
\begin{tabular}{l*{3}{c}}
\toprule
                                        &\multicolumn{1}{c}{(1)}&\multicolumn{1}{c}{(2)}&\multicolumn{1}{c}{(3)}\\
                                        &\multicolumn{1}{c}{HoG Turnover}&\multicolumn{1}{c}{HoS Turnover}&\multicolumn{1}{c}{L.H. Turnover}\\
\midrule
De Facto CBI                            &                  &                  &                  \\
No Turnover                             &   0.0734\sym{*}  &   0.0356         &    0.119\sym{**} \\
                                        &   (2.23)         &   (1.30)         &   (3.19)         \\
\addlinespace
Half Turnover                           & -0.00756\sym{*}  & -0.00655         &  -0.0296\sym{**} \\
                                        &  (-2.02)         &  (-1.24)         &  (-3.05)         \\
\addlinespace
Full Turnover                           &  -0.0658\sym{*}  &  -0.0290         &  -0.0890\sym{**} \\
                                        &  (-2.23)         &  (-1.31)         &  (-3.14)         \\
\midrule
Fixed Rate                              &                  &                  &                  \\
No Turnover                             &  0.00384         &  0.00473         & -0.00440         \\
                                        &   (1.32)         &   (1.93)         &  (-1.19)         \\
\addlinespace
Half Turnover                           &-0.000396         &-0.000870         &  0.00110         \\
                                        &  (-1.27)         &  (-1.87)         &   (1.18)         \\
\addlinespace
Full Turnover                           & -0.00345         & -0.00386         &  0.00331         \\
                                        &  (-1.32)         &  (-1.92)         &   (1.19)         \\
\midrule
Observations                            &     1651         &     1651         &     1334         \\
\bottomrule
\multicolumn{4}{l}{\footnotesize \textit{t} statistics in parentheses}\\
\multicolumn{4}{l}{\footnotesize \sym{*} \(p<0.05\), \sym{**} \(p<0.01\), \sym{***} \(p<0.001\)}\\
\end{tabular}
\end{table}

        }
    \end{frame}

    \begin{frame}
        \frametitle{Panel Logit (binary instability event variable) Mean Marginal Effects}
        \begin{itemize}
            \item Fixed effects
            \item More evidence that de jure CBI increases political instability
            \item Fixed exchange rate (low RR rate classification) increases pol. instability, but very small effect size
        \end{itemize}
    \end{frame}

    \begin{frame}
        \frametitle{Binary Instability Event Logit, Mean Marginal Effects}
        {
            \let\oldcentering\centering
            \renewcommand\centering{\tiny\oldcentering}
            \begin{table}[htbp]\centering
\def\sym#1{\ifmmode^{#1}\else\(^{#1}\)\fi}
\caption{De Jure CBI, Instability Event Panel Logit, Fixed Effects and Clustered Standard Errors, Mean Marginal Effects \label{margsJustBinInstabEventDJ}}
\begin{tabular}{l*{1}{c}}
\hline\hline
                    &\multicolumn{1}{c}{(1)}\\
                    &\multicolumn{1}{c}{} \\
\hline
De Jure CBI (CNW    &       0.376\sym{***}\\
Index)              &     (12.93)         \\
[1em]
Exchange Rate       &     0.00227\sym{**} \\
Classification (RR inverted, higher = more fixed)&      (2.99)         \\
\hline
Observations        &        3912         \\
\hline\hline
\multicolumn{2}{l}{\footnotesize \textit{t} statistics in parentheses}\\
\multicolumn{2}{l}{\footnotesize \sym{*} \(p<0.05\), \sym{**} \(p<0.01\), \sym{***} \(p<0.001\)}\\
\end{tabular}
\end{table}

        }
    \end{frame}

    \begin{frame}
        \frametitle{Binary Instability Event Logit, Mean Marginal Effects}
        {
            \let\oldcentering\centering
            \renewcommand\centering{\tiny\oldcentering}
            \begin{table}[htbp]\centering
\def\sym#1{\ifmmode^{#1}\else\(^{#1}\)\fi}
\caption{De Facto CBI, Instability Event Panel Logit, Fixed Effects and Clustered Standard Errors, Mean Marginal Effects \label{margsJustBinInstabEventDJ}}
\begin{tabular*}{\linewidth}{@{\hskip\tabcolsep\extracolsep\fill}l*{1}{c}}
\hline\hline
                &\multicolumn{1}{c}{(1)}\\
                &\multicolumn{1}{c}{}\\
\hline
(Lack of) Irregular CB Governor Turnover (higher = more de facto CBI)&   0.0282         \\
                &   (1.18)         \\
[1em]
Exchange Rate Classification (RR inverted, higher = more fixed)&   0.0152\sym{***}\\
                &   (6.71)         \\
\hline
Observations    &     4163         \\
\hline\hline
\multicolumn{2}{l}{\footnotesize \textit{t} statistics in parentheses}\\
\multicolumn{2}{l}{\footnotesize \sym{*} \(p<0.05\), \sym{**} \(p<0.01\), \sym{***} \(p<0.001\)}\\
\end{tabular*}
\end{table}

        }
    \end{frame}

    \begin{frame}
        \frametitle{IV1: Tertiary Ed Enrollment (CBI), Aggregate GDP (Fixed Rate)}
        \begin{itemize}
            \item Good first stages
            \item Poor exclusion restrictions for political stability, better ones for electoral stability/turnover
            \item De jure CBI now increases lower chamber turnover, but no longer HOS; strange sign for WB stability
            \item Fixed rates appear to increase instability
            \item De facto CBI more or less insignificant
        \end{itemize}
    \end{frame}

    \begin{frame}
        \frametitle{Tertiary Education and Aggregate GDP Instruments}
        {
            \let\oldcentering\centering
            \renewcommand\centering{\tiny\oldcentering}
            \begin{table}[htbp]\centering
\def\sym#1{\ifmmode^{#1}\else\(^{#1}\)\fi}
\caption{Instruments of Tertiary Education Enrollment Rate and Aggregate GDP, Robust Standard Errors \label{ifivs}}
\begin{tabular}{l*{5}{c}}
\toprule
                                        &\multicolumn{1}{c}{(1)}&\multicolumn{1}{c}{(2)}&\multicolumn{1}{c}{(3)}&\multicolumn{1}{c}{(4)}&\multicolumn{1}{c}{(5)}\\
                                        &\multicolumn{1}{c}{Head of Govt. Turnover}&\multicolumn{1}{c}{Head of State Turnover}&\multicolumn{1}{c}{Lower House Turnover}&\multicolumn{1}{c}{WB Political Stability (Absence of Violence)}&\multicolumn{1}{c}{Instability Event Indicator}\\
\midrule
De Jure CBI (CNW Index)                 &    0.629         &   -0.478         &    0.847\sym{*}  &    6.976\sym{***}&    0.835\sym{***}\\
                                        &   (1.55)         &  (-1.42)         &   (1.97)         &  (13.27)         &   (4.30)         \\
\addlinespace
Exchange Rate Classification (RR        & -0.00669         &   0.0171         &   0.0266         &  -0.0865\sym{**} &  -0.0295         \\
inverted, higher = more fixed)          &  (-0.19)         &   (0.51)         &   (0.76)         &  (-2.84)         &  (-1.66)         \\
\addlinespace
Constant                                &    0.401         &    0.576\sym{*}  &   0.0636         &   -3.422\sym{***}&    0.292         \\
                                        &   (1.28)         &   (2.01)         &   (0.22)         &  (-9.22)         &   (1.65)         \\
\midrule
Observations                            &      851         &      851         &      686         &     1865         &     2047         \\
\bottomrule
\multicolumn{6}{l}{\footnotesize \textit{t} statistics in parentheses}\\
\multicolumn{6}{l}{\footnotesize \sym{*} \(p<0.05\), \sym{**} \(p<0.01\), \sym{***} \(p<0.001\)}\\
\end{tabular}
\end{table}

        }
    \end{frame}

    \begin{frame}
        \frametitle{Tertiary Education and Aggregate GDP Instruments}
        {
            \let\oldcentering\centering
            \renewcommand\centering{\tiny\oldcentering}
                                &ifivs2_v2elturnhog&ifivs2_v2elturnhos&ifivs2_v2eltvrig&ifivs2_e_wbgi_pve&ifivs2_instabEvent\\
                    &           b&           b&           b&           b&           b\\
(Lack of) Irregular CB Governor Turnover (higher = more de facto CBI)&    1.295211&    -.626102&    2.070568&    39.46744&   -18.00985\\
Exchange Rate Classification (RR inverted, higher = more fixed)&    .0151971&   -.0101352&    .0864062&    .5809117&   -.1314777\\
_cons               &   -.5376844&    1.085258&   -1.708465&   -40.72482&    17.29254\\

        }
    \end{frame}

    \begin{frame}
        \frametitle{IV2: Population Share Social Science/Business Grads (CBI), Agg GDP (Fixed Rates)}
        \begin{itemize}
            \item Better Exclusion Restriction
            \item Very limited data but strong result for de jure CBI and political instability
        \end{itemize}
    \end{frame}

    \begin{frame}
        \frametitle{Population Share Social Science/Business Grads and Agg GDP Instruments}
        {
            \let\oldcentering\centering
            \renewcommand\centering{\tiny\oldcentering}
            \begin{table}[htbp]\centering
\def\sym#1{\ifmmode^{#1}\else\(^{#1}\)\fi}
\caption{Instruments of Social Science/Business Graduates Population Share and Aggregate GDP, Robust Standard Errors \label{ifivs3}}
\begin{tabular}{l*{5}{c}}
\toprule
                                        &\multicolumn{1}{c}{(1)}&\multicolumn{1}{c}{(2)}&\multicolumn{1}{c}{(3)}&\multicolumn{1}{c}{(4)}&\multicolumn{1}{c}{(5)}\\
                                        &\multicolumn{1}{c}{HoG Turnover}&\multicolumn{1}{c}{HoS Turnover}&\multicolumn{1}{c}{L. H. Turnover}&\multicolumn{1}{c}{WB Pol. Stability}&\multicolumn{1}{c}{Instab. Event}\\
\midrule
De Jure CBI                             &    44.33         &    14.48         &   -22.04         &   -19.44         &    2.704\sym{***}\\
                                        &   (0.49)         &   (0.47)         &  (-0.48)         &  (-0.24)         &   (4.11)         \\
\addlinespace
Fixed Rate                              &   -1.277         &   -0.422         &    0.704         &    0.722         &   -0.129         \\
                                        &  (-0.47)         &  (-0.44)         &   (0.51)         &   (0.27)         &  (-1.60)         \\
\addlinespace
Constant                                &   -19.38         &   -6.144         &    10.39         &    8.414         &                  \\
                                        &  (-0.50)         &  (-0.46)         &   (0.52)         &   (0.25)         &                  \\
\midrule
Observations                            &       20         &       20         &       17         &       53         &       12         \\
\bottomrule
\multicolumn{6}{l}{\footnotesize \textit{t} statistics in parentheses}\\
\multicolumn{6}{l}{\footnotesize \sym{*} \(p<0.05\), \sym{**} \(p<0.01\), \sym{***} \(p<0.001\)}\\
\end{tabular}
\end{table}

        }
    \end{frame}

    \begin{frame}
        \frametitle{Population Share Social Science/Business Grads and Agg GDP Instruments}
        {
            \let\oldcentering\centering
            \renewcommand\centering{\tiny\oldcentering}
                                &ifivs4_v2elturnhog&ifivs4_v2elturnhos&ifivs4_v2eltvrig&ifivs4_e_wbgi_pve\\
                    &           b&           b&           b&           b\\
(Lack of) Irregular CB Governor Turnover (higher = more de facto CBI)&   -18.95369&   -5.277986&     19.3662&   -7.488099\\
Exchange Rate Classification (RR inverted, higher = more fixed)&     .012854&   -.0133111&    .1310775&    .0659122\\
_cons               &    19.38204&    5.798952&   -19.05845&    7.212124\\

        }
    \end{frame}

    \begin{frame}
        \frametitle{Just Aggregate GDP for Fixed Rates}
        \begin{itemize}
            \item Clearer case for fixed rates decreasing pol and electoral stability (PBC)
            \item Note on exclusion restriction: still an imperfect case
            \begin{itemize}
                \item Agg GDP proxies for economy size (optimum currency area)
                \item Arguably not as connected to GDP per capita to stability
            \end{itemize}
        \end{itemize}
    \end{frame}

    \begin{frame}
        \frametitle{Aggregate GDP Instrument for Fixed Rates}
        {
            \let\oldcentering\centering
            \renewcommand\centering{\tiny\oldcentering}
            {
\def\sym#1{\ifmmode^{#1}\else\(^{#1}\)\fi}
\begin{tabular*}{\linewidth}{@{\hskip\tabcolsep\extracolsep\fill}l*{2}{c}}
\toprule
                &\multicolumn{1}{c}{(1)}&\multicolumn{1}{c}{(2)}\\
                &\multicolumn{1}{c}{Lower House Turnover}&\multicolumn{1}{c}{WB Political Stability (Absence of Violence)}\\
\midrule
Exchange Rate Classification (RR inverted, higher = more fixed)&   0.0779\sym{***}&   -0.257\sym{***}\\
                &   (3.35)         &  (-4.13)         \\
\addlinespace
Constant        &   0.0991         &    1.992\sym{***}\\
                &   (0.58)         &   (4.16)         \\
\midrule
Observations    &      835         &      437         \\
\bottomrule
\multicolumn{3}{l}{\footnotesize \textit{t} statistics in parentheses}\\
\multicolumn{3}{l}{\footnotesize \sym{*} \(p<0.05\), \sym{**} \(p<0.01\), \sym{***} \(p<0.001\)}\\
\end{tabular*}
}

        }
    \end{frame}

    \begin{frame}
        \frametitle{Table of Lags (see paper)}
        \begin{itemize}
            \item Additional observations for the longer term:
            \begin{itemize}
                \item T-3 sees strongest de jure CBI political instability impact
                \item T-6, T-8 de jure CBI increases pol instability. T-8 reduces HOG turnover (electoral instability) (similar to Clark, Golder, and Poast).
                \item Fixed rates increase instability in the same T-6 and up range
                \item De facto CBI not very significant
                \item Similar results with lagged ordinal logit specification, though de facto CBI more significant in reducing L.H. turnover
            \end{itemize}
        \end{itemize}
    \end{frame}

    \begin{frame}
        \frametitle{Institutional Interaction Terms}
        \begin{itemize}
            \item De jure CBI and fixed rates jointly grow political instability
            \item Signs mixed for other kinds of instability
            \item De facto CBI and fixed rates in combination somewhat increase instability relative to individually
            \item Pseudo Mundell-Fleming trilemma and PBCs: more difficult to manage the economy
            \begin{itemize}
                \item Is CBI a good representation of "domestic monetary autonomy"?
                \item See appendix for test with capital controls explicitly
            \end{itemize}
        \end{itemize}
    \end{frame}

    \begin{frame}
        \frametitle{Summary}
        \begin{itemize}
            \item De jure CBI generally decreases (esp. pol) stability, suggesting limits on PBCs
            \item Unclear sign for de facto CBI though it appears to increase stability if anything
            \item Fixed rates mostly appear to increase stability in fixed effects regressions, but the sign flips in more robust models (IV, lags)
            \item Combinations/interactions of commitment institutions more destabilizing
            \item Commitment institutions more often politically costly relative to previous literature
            \item Robust results
            \begin{itemize}
                \item Not covered: institutional controls for federalism and corporatism do not affect signs or cause large changes in effects, expected results on HOS = HOG and less expected ones for legislative power in practice, interactions with democracy do not behave as expected, capital account openness generally destabilizing, somewhat mitigated with interactions with fixed rates and de facto CBI
            \end{itemize}
        \end{itemize}
    \end{frame}

    \begin{frame}
        \frametitle{Questions/future directions}
        \begin{itemize}
            \item More complex theory for de jure versus de facto CBI puzzle, importance of credible commitment
            \item Diverging predictions for Head of Government, Head of State, Lower House Turnover
            \begin{itemize}
                \item HOS and Lower House seem to have strongest relationships
            \end{itemize}
            \item Endogenous elections
            \item Dynamic panel (A-Bond)?
            \item Ordinal logit regression with IV (different procedure)
        \end{itemize}
    \end{frame}

    \begin{frame}
        \centering Further Explorations
    \end{frame}

    \begin{frame}
        \frametitle{Controls}
        \begin{itemize}
            \item Regional government exists and has autonomy and authority, checks and balances/horizontal accountability
            \item Not strictly necessary
                \begin{itemize}
                    \item Many items already included in FEs
                    \item No sign flips for main variables
                \end{itemize}
            \item Omitted: Corporatism
            \item The controls themselves are often significant and somewhat interesting
        \end{itemize}
    \end{frame}

    \begin{frame}
        \frametitle{Controls Excluding Corporatism}
        {
            \let\oldcentering\centering
            \renewcommand\centering{\tiny\oldcentering}
            \begin{table}[htbp]\centering
\def\sym#1{\ifmmode^{#1}\else\(^{#1}\)\fi}
\caption{All Controls Excluding Corporatism, Fixed Effects and Clustered Standard Errors \label{nccmultIndFEDJ}}
\begin{tabular}{l*{5}{c}}
\toprule
                                        &\multicolumn{1}{c}{(1)}&\multicolumn{1}{c}{(2)}&\multicolumn{1}{c}{(3)}&\multicolumn{1}{c}{(4)}&\multicolumn{1}{c}{(5)}\\
                                        &\multicolumn{1}{c}{HoG Turnover}&\multicolumn{1}{c}{HoS Turnover}&\multicolumn{1}{c}{L. H. Turnover}&\multicolumn{1}{c}{WB Pol. Stability}&\multicolumn{1}{c}{Instab. Event}\\
\midrule
De Jure CBI                             &    0.181         &    0.151         &    0.481         &   -0.531         &    0.961\sym{***}\\
                                        &   (0.62)         &   (0.70)         &   (1.25)         &  (-1.98)         &   (5.33)         \\
\addlinespace
Fixed Rate                              & -0.00641         &  -0.0356\sym{***}&  0.00257         &-0.000489         &   0.0232\sym{*}  \\
                                        &  (-0.48)         &  (-3.93)         &   (0.15)         &  (-0.05)         &   (2.49)         \\
\addlinespace
Reg. Govt. Exists                       &    0.863\sym{***}&0.0000816         &    1.010\sym{***}&    0.107         &   -0.221\sym{*}  \\
                                        &   (3.65)         &   (0.00)         &   (3.50)         &   (1.98)         &  (-2.22)         \\
\addlinespace
Horiz. Acctability                      &    0.390\sym{**} &    0.371\sym{**} &    0.220         &   0.0639         &    0.100\sym{*}  \\
                                        &   (3.30)         &   (3.38)         &   (1.85)         &   (0.56)         &   (2.20)         \\
\addlinespace
Checks and Balances                     &  -0.0126         &  -0.0392         &  0.00165         &  0.00951         &  0.00762         \\
                                        &  (-0.31)         &  (-1.40)         &   (0.04)         &   (0.75)         &   (0.63)         \\
\addlinespace
Autonomous Regions                      &   -0.714         &  -0.0764         &   -1.274\sym{***}&   -0.359\sym{***}&  -0.0416         \\
                                        &  (-1.37)         &  (-0.58)         &  (-4.10)         &  (-7.85)         &  (-0.69)         \\
\addlinespace
State Govt. Auth.                       &    0.306         &   0.0825         &    0.465         &        0         &  -0.0651         \\
                                        &   (0.40)         &   (1.19)         &   (1.65)         &      (.)         &  (-1.28)         \\
\addlinespace
Constant                                &   -0.317         &    0.522\sym{**} &   -0.676\sym{*}  &    0.168         &   -0.164         \\
                                        &  (-0.73)         &   (2.67)         &  (-2.35)         &   (0.95)         &  (-1.46)         \\
\midrule
Observations                            &      483         &      483         &      415         &      780         &     1389         \\
\bottomrule
\multicolumn{6}{l}{\footnotesize \textit{t} statistics in parentheses}\\
\multicolumn{6}{l}{\footnotesize \sym{*} \(p<0.05\), \sym{**} \(p<0.01\), \sym{***} \(p<0.001\)}\\
\end{tabular}
\end{table}

        }
    \end{frame}

    \begin{frame}
        \frametitle{Controls Excluding Corporatism}
        {
            \let\oldcentering\centering
            \renewcommand\centering{\tiny\oldcentering}
            {
\def\sym#1{\ifmmode^{#1}\else\(^{#1}\)\fi}
\begin{tabular}{l*{5}{c}}
\hline\hline
                    &\multicolumn{1}{c}{(1)}&\multicolumn{1}{c}{(2)}&\multicolumn{1}{c}{(3)}&\multicolumn{1}{c}{(4)}&\multicolumn{1}{c}{(5)}\\
                    &\multicolumn{1}{c}{Head of Govt. Turnover}&\multicolumn{1}{c}{Head of State Turnover}&\multicolumn{1}{c}{Lower House Turnover}&\multicolumn{1}{c}{WB Political Stability (Absence of Violence)}&\multicolumn{1}{c}{Instability Event Indicator}\\
\hline
(Lack of) Irregular &      -0.264\sym{*}  &      -0.119         &      -0.321\sym{*}  &      0.0570         &      0.0307         \\
CB Governor Turnover (higher = more de facto CBI)&     (-2.39)         &     (-1.13)         &     (-2.57)         &      (1.44)         &      (0.98)         \\
[1em]
Exchange Rate       &    -0.00661         &     -0.0207\sym{*}  &     0.00415         &   -0.000246         &      0.0311\sym{***}\\
Classification (RR inverted, higher = more fixed)&     (-0.56)         &     (-2.16)         &      (0.24)         &     (-0.03)         &      (3.53)         \\
[1em]
Regional government &       0.681\sym{**} &      0.0312         &       0.985\sym{***}&      0.0731         &     -0.0622         \\
exists              &      (2.75)         &      (0.33)         &      (5.03)         &      (0.68)         &     (-0.36)         \\
[1em]
Horizontal          &       0.306\sym{**} &       0.308\sym{**} &       0.223         &      0.0329         &       0.133\sym{*}  \\
accountability index&      (3.17)         &      (3.24)         &      (1.81)         &      (0.34)         &      (2.36)         \\
[1em]
Checks and Balances &     -0.0415         &     -0.0507         &    -0.00753         &     0.01000         &    -0.00346         \\
                    &     (-1.22)         &     (-1.74)         &     (-0.20)         &      (0.61)         &     (-0.27)         \\
[1em]
Autonomous Regions  &      -0.553         &     -0.0437         &      -1.206\sym{**} &      -0.302\sym{***}&      0.0203         \\
                    &     (-1.10)         &     (-0.64)         &     (-3.16)         &     (-7.57)         &      (0.23)         \\
[1em]
State Government    &       0.308         &      0.0861         &       0.615\sym{*}  &           0         &       0.123         \\
Authority over Taxing, Spending, or Legislating&      (0.38)         &      (1.41)         &      (2.52)         &         (.)         &      (1.45)         \\
[1em]
Constant            &       0.322         &       0.651\sym{***}&      -0.134         &      -0.192         &      0.0226         \\
                    &      (0.71)         &      (4.73)         &     (-0.56)         &     (-1.12)         &      (0.15)         \\
\hline
Observations        &         563         &         563         &         477         &         993         &        1416         \\
\hline\hline
\multicolumn{6}{l}{\footnotesize \textit{t} statistics in parentheses}\\
\multicolumn{6}{l}{\footnotesize \sym{*} \(p<0.05\), \sym{**} \(p<0.01\), \sym{***} \(p<0.001\)}\\
\end{tabular}
}

        }
    \end{frame}

    \begin{frame}
        \frametitle{HOS = HOG?}
        \begin{itemize}
            \item V2exhoshog is an indicator for whether HOS and HOG are the same person
            \item De jure CBI has more of an impact on both positions when they are the same individual
            \item However, fixed rates only affect HoS turnover when the head of state and head of government are not the same individual
            \item Overall, in most cases, effects occur when HoS = HoG (and definitely HoS effects mostly appear when this is the case. See paper for lagged interaction term analysis.)
            \begin{itemize}
                \item Logically more accountability for HoG, unless they are the same person
                \item Direct accountability under presidentialism?
            \end{itemize}
        \end{itemize}
    \end{frame}

    \begin{frame}
        \frametitle{HOS = HOG, Tertiary Education Instrument}
        {
            \let\oldcentering\centering
            \renewcommand\centering{\tiny\oldcentering}
                                &hoshogfivs_v2elturnhog&hoshogfivs_v2elturnhos&hoshogfivs_v2eltvrig&hoshogfivs_e_wbgi_pve&hoshogfivs_instabEvent\\
                    &           b&           b&           b&           b&           b\\
De Jure CBI (CNW Index)&    1.912017&    1.881619&    1.047083&    3.977786&    .7091629\\
Exchange Rate Classification (RR inverted, higher = more fixed)&   -.0142326&    -.014013&    .0003052&   -.1638646&   -.0122039\\
_cons               &   -.2798882&   -.2677375&    .0264392&   -1.484421&    .1788975\\

        }
    \end{frame}

    \begin{frame}
        \frametitle{HOS = HOG, Tertiary Education Instrument}
        {
            \let\oldcentering\centering
            \renewcommand\centering{\tiny\oldcentering}
            \begin{table}[htbp]\centering
\def\sym#1{\ifmmode^{#1}\else\(^{#1}\)\fi}
\caption{HOS = HOG, Tertiary Education Instrument, Robust Standard Errors \label{hoshogfivs2}}
\begin{tabular}{l*{5}{c}}
\toprule
                                        &\multicolumn{1}{c}{(1)}&\multicolumn{1}{c}{(2)}&\multicolumn{1}{c}{(3)}&\multicolumn{1}{c}{(4)}&\multicolumn{1}{c}{(5)}\\
                                        &\multicolumn{1}{c}{HoG Turnover}&\multicolumn{1}{c}{HoS Turnover}&\multicolumn{1}{c}{L. H. Turnover}&\multicolumn{1}{c}{WB Pol. Stability}&\multicolumn{1}{c}{Instab. Event}\\
\midrule
De facto CBI                            &    14.23         &    17.59         &    3.692         &   -26.08         &   -1.598         \\
                                        &   (0.39)         &   (0.40)         &   (0.39)         &  (-0.92)         &  (-1.81)         \\
\addlinespace
Fixed Rate                              &    0.196         &    0.241         &   0.0972         &   -0.320         &  -0.0238         \\
                                        &   (0.41)         &   (0.41)         &   (0.74)         &  (-0.89)         &  (-0.85)         \\
\addlinespace
Constant                                &   -12.74         &   -15.91         &   -3.186         &    25.22         &    1.987\sym{*}  \\
                                        &  (-0.38)         &  (-0.38)         &  (-0.36)         &   (0.92)         &   (2.30)         \\
\midrule
Observations                            &      323         &      323         &      279         &      760         &      734         \\
\bottomrule
\multicolumn{6}{l}{\footnotesize \textit{t} statistics in parentheses}\\
\multicolumn{6}{l}{\footnotesize \sym{*} \(p<0.05\), \sym{**} \(p<0.01\), \sym{***} \(p<0.001\)}\\
\end{tabular}
\end{table}

        }
    \end{frame}

    \begin{frame}
        \frametitle{HOS NOT HOG, Tertiary Education Instrument}
        {
            \let\oldcentering\centering
            \renewcommand\centering{\tiny\oldcentering}
            \begin{table}[htbp]\centering
\def\sym#1{\ifmmode^{#1}\else\(^{#1}\)\fi}
\caption{HOS NOT HOG, Tertiary Education Instrument, Robust Standard Errors \label{NOhoshogIfivs}}
\begin{tabular}{l*{5}{c}}
\toprule
                                        &\multicolumn{1}{c}{(1)}&\multicolumn{1}{c}{(2)}&\multicolumn{1}{c}{(3)}&\multicolumn{1}{c}{(4)}&\multicolumn{1}{c}{(5)}\\
                                        &\multicolumn{1}{c}{HoG Turnover}&\multicolumn{1}{c}{HoS Turnover}&\multicolumn{1}{c}{L. House Turnover}&\multicolumn{1}{c}{WB Pol. Stability}&\multicolumn{1}{c}{Instab. Event}\\
\midrule
De Jure CBI (CNW Index)                 &    0.196         &  -0.0412         &    0.471         &    7.754\sym{***}&   0.0619         \\
                                        &   (0.43)         &  (-0.12)         &   (1.05)         &   (6.82)         &   (0.12)         \\
\addlinespace
Exchange Rate Classification (RR        &  -0.0306         &   0.0661\sym{**} &   0.0482         &    0.190         &  -0.0965         \\
inverted, higher = more fixed)          &  (-0.69)         &   (2.64)         &   (1.14)         &   (1.37)         &  (-1.87)         \\
\addlinespace
Constant                                &    0.859         &   -0.237         &    0.141         &   -6.252\sym{***}&    1.284         \\
                                        &   (1.70)         &  (-0.82)         &   (0.30)         &  (-3.41)         &   (1.91)         \\
\midrule
Observations                            &      560         &      560         &      439         &     1211         &     1282         \\
\bottomrule
\multicolumn{6}{l}{\footnotesize \textit{t} statistics in parentheses}\\
\multicolumn{6}{l}{\footnotesize \sym{*} \(p<0.05\), \sym{**} \(p<0.01\), \sym{***} \(p<0.001\)}\\
\end{tabular}
\end{table}

        }
    \end{frame}

    \begin{frame}
        \frametitle{HOS NOT HOG, Tertiary Education Instrument}
        {
            \let\oldcentering\centering
            \renewcommand\centering{\tiny\oldcentering}
            {
\def\sym#1{\ifmmode^{#1}\else\(^{#1}\)\fi}
\begin{tabular}{l*{5}{c}}
\toprule
                                        &\multicolumn{1}{c}{(1)}&\multicolumn{1}{c}{(2)}&\multicolumn{1}{c}{(3)}&\multicolumn{1}{c}{(4)}&\multicolumn{1}{c}{(5)}\\
                                        &\multicolumn{1}{c}{Head of Govt. Turnover}&\multicolumn{1}{c}{Head of State Turnover}&\multicolumn{1}{c}{Lower House Turnover}&\multicolumn{1}{c}{WB Political Stability (Absence of Violence)}&\multicolumn{1}{c}{Instability Event Indicator}\\
\midrule
(Lack of) Irregular CB Governor Turnover&     0.475         &     0.535         &     1.872         &     145.7         &     40.28         \\
(higher = more de facto CBI)            &    (0.31)         &    (0.41)         &    (0.97)         &    (0.41)         &    (0.05)         \\
\addlinespace
Exchange Rate Classification (RR        &   -0.0120         &    0.0841         &     0.110         &     4.603         &    -0.224         \\
inverted, higher = more fixed)          &   (-0.16)         &    (1.62)         &    (1.11)         &    (0.39)         &   (-0.12)         \\
\addlinespace
Constant                                &     0.404         &    -0.893         &    -1.730         &    -176.0         &    -33.39         \\
                                        &    (0.21)         &   (-0.57)         &   (-0.69)         &   (-0.40)         &   (-0.05)         \\
\midrule
Observations                            &       639         &       639         &       509         &      1476         &      1277         \\
\bottomrule
\multicolumn{6}{l}{\footnotesize \textit{t} statistics in parentheses}\\
\multicolumn{6}{l}{\footnotesize \sym{*} \(p<0.05\), \sym{**} \(p<0.01\), \sym{***} \(p<0.001\)}\\
\end{tabular}
}

        }
    \end{frame}

    \begin{frame}
        \frametitle{Legislative Power in Practice}
        \begin{itemize}
            \item Examining robust results, effects on lower house turnover are occasionally more in line with expectations when it has power in practice
            \item Not clear evidence of a matching of power and accountability
            \item See paper for lagged interaction term analysis
        \end{itemize}
    \end{frame}

    \begin{frame}
        \frametitle{Democracy/Nondemocracy}
        \begin{itemize}
            \item Classification based on Polity IV scores
            \item Tested:
            \begin{itemize}
                \item De jure versus de facto CBI matters more in democracies due to rule of law
                \item Electoral versus political turnover/instability matters more in democracies
            \end{itemize}
            \item Not really any consistent pattern of major differences. All items generally more significant in democracies
            \item See paper for lagged interaction term analysis, which somewhat challenges the last point
        \end{itemize}
    \end{frame}

    \begin{frame}
        \frametitle{Democracies, Tertiary Education Instrument}
        {
            \let\oldcentering\centering
            \renewcommand\centering{\tiny\oldcentering}
            {
\def\sym#1{\ifmmode^{#1}\else\(^{#1}\)\fi}
\begin{tabular}{l*{5}{c}}
\hline\hline
                    &\multicolumn{1}{c}{(1)}&\multicolumn{1}{c}{(2)}&\multicolumn{1}{c}{(3)}&\multicolumn{1}{c}{(4)}&\multicolumn{1}{c}{(5)}\\
                    &\multicolumn{1}{c}{Head of Govt. Turnover}&\multicolumn{1}{c}{Head of State Turnover}&\multicolumn{1}{c}{Lower House Turnover}&\multicolumn{1}{c}{WB Political Stability (Absence of Violence)}&\multicolumn{1}{c}{Instability Event Indicator}\\
\hline
De Jure CBI (CNW    &      -0.453         &      -1.933\sym{***}&       0.128         &       6.173\sym{***}&       1.032\sym{***}\\
Index)              &     (-0.77)         &     (-3.32)         &      (0.21)         &     (10.70)         &      (4.25)         \\
[1em]
Exchange Rate       &      0.0486         &      0.0754         &      0.0554         &      -0.118\sym{***}&     -0.0471\sym{**} \\
Classification (RR inverted, higher = more fixed)&      (1.21)         &      (1.85)         &      (1.37)         &     (-4.41)         &     (-2.70)         \\
[1em]
Constant            &       0.737\sym{*}  &       1.103\sym{***}&       0.326         &      -2.778\sym{***}&       0.271         \\
                    &      (2.54)         &      (3.79)         &      (1.20)         &     (-8.71)         &      (1.83)         \\
\hline
Observations        &         615         &         615         &         510         &        1297         &        1387         \\
\hline\hline
\multicolumn{6}{l}{\footnotesize \textit{t} statistics in parentheses}\\
\multicolumn{6}{l}{\footnotesize \sym{*} \(p<0.05\), \sym{**} \(p<0.01\), \sym{***} \(p<0.001\)}\\
\end{tabular}
}

        }
    \end{frame}

    \begin{frame}
        \frametitle{Democracies, Tertiary Education Instrument}
        {
            \let\oldcentering\centering
            \renewcommand\centering{\tiny\oldcentering}
                                &demIfivs2_v2elturnhog&demIfivs2_v2elturnhos&demIfivs2_v2eltvrig&demIfivs2_e_wbgi_pve&demIfivs2_instabEvent\\
                    &           b&           b&           b&           b&           b\\
(Lack of) Irregular CB Governor Turnover (higher = more de facto CBI)&   -.8419318&   -2.289557&    .4521762&    23.30232&    9.328988\\
Exchange Rate Classification (RR inverted, higher = more fixed)&    .0119517&   -.0157008&    .0774808&    .2519216&    .0353563\\
_cons               &    1.492012&    2.672976&   -.1336206&   -23.09345&   -7.906109\\

        }
    \end{frame}

    \begin{frame}
        \frametitle{Nondemocracies, Tertiary Education Instrument}
        {
            \let\oldcentering\centering
            \renewcommand\centering{\tiny\oldcentering}
            {
\def\sym#1{\ifmmode^{#1}\else\(^{#1}\)\fi}
\begin{tabular}{l*{5}{c}}
\hline\hline
                &\multicolumn{1}{c}{(1)}&\multicolumn{1}{c}{(2)}&\multicolumn{1}{c}{(3)}&\multicolumn{1}{c}{(4)}&\multicolumn{1}{c}{(5)}\\
                &\multicolumn{1}{c}{Head of Govt. Turnover}&\multicolumn{1}{c}{Head of State Turnover}&\multicolumn{1}{c}{Lower House Turnover}&\multicolumn{1}{c}{WB Political Stability (Absence of Violence)}&\multicolumn{1}{c}{Instability Event Indicator}\\
\hline
De Jure CBI (CNW Index)&   -1.937         &   -0.295         &    1.119         &    15.47         &    5.034         \\
                &  (-0.99)         &  (-0.30)         &   (0.63)         &   (1.49)         &   (0.71)         \\
[1em]
Exchange Rate Classification (RR inverted, higher = more fixed)&    0.110\sym{*}  &   0.0316         &   0.0887         &   -1.014         &   -0.183         \\
                &   (2.14)         &   (0.99)         &   (1.42)         &  (-1.76)         &  (-0.97)         \\
[1em]
Constant        &    0.172         &  -0.0255         &   -0.780         &    1.248         &   -0.207         \\
                &   (0.22)         &  (-0.06)         &  (-1.01)         &   (0.62)         &  (-0.13)         \\
\hline
Observations    &      194         &      194         &      146         &      476         &      602         \\
\hline\hline
\multicolumn{6}{l}{\footnotesize \textit{t} statistics in parentheses}\\
\multicolumn{6}{l}{\footnotesize \sym{*} \(p<0.05\), \sym{**} \(p<0.01\), \sym{***} \(p<0.001\)}\\
\end{tabular}
}

        }
    \end{frame}

    \begin{frame}
        \frametitle{Nondemocracies, Tertiary Education Instrument}
        {
            \let\oldcentering\centering
            \renewcommand\centering{\tiny\oldcentering}
            {
\def\sym#1{\ifmmode^{#1}\else\(^{#1}\)\fi}
\begin{tabular}{l*{5}{c}}
\toprule
                                        &\multicolumn{1}{c}{(1)}&\multicolumn{1}{c}{(2)}&\multicolumn{1}{c}{(3)}&\multicolumn{1}{c}{(4)}&\multicolumn{1}{c}{(5)}\\
                                        &\multicolumn{1}{c}{Head of Govt. Turnover}&\multicolumn{1}{c}{Head of State Turnover}&\multicolumn{1}{c}{Lower House Turnover}&\multicolumn{1}{c}{WB Political Stability (Absence of Violence)}&\multicolumn{1}{c}{Instability Event Indicator}\\
\midrule
(Lack of) Irregular CB Governor Turnover&     76.68         &    -33.62         &     0.748         &    -1.311         &    -1.726         \\
(higher = more de facto CBI)            &    (0.05)         &   (-0.08)         &    (0.19)         &   (-0.41)         &   (-0.68)         \\
\addlinespace
Exchange Rate Classification (RR        &     2.058         &    -0.588         &     0.108         &    -0.559\sym{***}&   -0.0535         \\
inverted, higher = more fixed)          &    (0.05)         &   (-0.08)         &    (0.72)         &   (-3.58)         &   (-0.69)         \\
\addlinespace
Constant                                &    -85.72         &     34.88         &    -1.102         &     5.928\sym{*}  &     2.491         \\
                                        &   (-0.05)         &    (0.08)         &   (-0.24)         &    (2.20)         &    (1.25)         \\
\midrule
Observations                            &       186         &       187         &       146         &       522         &       566         \\
\bottomrule
\multicolumn{6}{l}{\footnotesize \textit{t} statistics in parentheses}\\
\multicolumn{6}{l}{\footnotesize \sym{*} \(p<0.05\), \sym{**} \(p<0.01\), \sym{***} \(p<0.001\)}\\
\end{tabular}
}

        }
    \end{frame}

    \begin{frame}
        \frametitle{Capital Account Openness Interactions}
        \begin{itemize}
            \item See paper for lagged interaction term analysis. Some results, particularly those for the ordinal logit model, agree with Bernhard et al. (2002)
            \item Interesting side result on the political optimality of Bretton Woods (closed capital account, fixed rates, de facto CBI)
        \end{itemize}
    \end{frame}

\end{document}
